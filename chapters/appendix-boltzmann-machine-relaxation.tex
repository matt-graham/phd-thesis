\chapter{Boltzmann machine relaxations}\label{app:boltzmann-machine-relaxation}

In this appendix we describe the continuous relaxation of the Boltzmann machine distribution introduced in \citep{zhang2012continuous} and derive some basic results which we use in the experiments with this model in Chapter \ref{ch:continuous-tempering}.

The \emph{Boltzmann machine distribution} on $\rvct{s} \in \lbrace -1,+1\rbrace^{D_B} = \set{S}$ is defined
\begin{equation}
\begin{split}
  \prob{\rvct{s}}(\vct{s}) &= 
  \frac{1}{Z_B} \exp\lpa\frac{1}{2}\vct{s}\tr\mtx{W}\vct{s} + \vct{s}\tr\vct{b}\rpa
  \\
  Z_B &= \sum_{\vct{s}\in\set{S}} \lpa\exp\lpa\frac{1}{2}\vct{s}\tr\mtx{W}\vct{s} + \vct{s}\tr\vct{b}\rpa\rpa.
\end{split}
\end{equation}
\noindent
We introduce an auxiliary real-valued vector random variable $\rvct{x}\in\reals^D$ with a Gaussian conditional distribution
\begin{equation}
  \pden{\rvct{x}|\rvct{s}}(\vct{x} \gvn \vct{s}) =
  \frac{1}{(2\pi)^{\nicefrac{D}{2}}} \exp\lpa 
    -\frac{1}{2} \lpa \vct{x} - \mtx{Q}\tr\vct{s} \rpa \tr \lpa \vct{x} - \mtx{Q}\tr\vct{s} \rpa 
  \rpa
\end{equation}
with $\mtx{Q}$ a $D_B \times D$ matrix such that $\mtx{Q}\mtx{Q}\tr = \mtx{W} + \mtx{D}$ for a diagonal $\mtx{D}$ such that $\mtx{W} + \mtx{D} \succeq 0$. In our experiments, based on the observation in \citep{zhang2012continuous} that minimising the maximum eigenvalue of $\mtx{W} + \mtx{D}$ decreases the maximal separation between the Gaussian components in the relaxation, we set $\mtx{D}$ as the solution to the semi-definite programme 
\begin{equation}
  \min_{\mtx{D}} \lpa \lambda_{\textsc{max}}\lpa \mtx{W} + \mtx{D} \rpa \rpa 
  : \mtx{W} + \mtx{D} \succeq 0
\end{equation}
where $\lambda_{\textsc{max}}$ denotes the maximal eigenvalue. In general the optimised $\mtx{W} + \mtx{D}$ lies on the semi-definite cone and so has rank less than $D_B$ hence a $\mtx{Q}$ can be found such that $D < D_B$.

\noindent
The resulting joint distribution on $(\rvct{x},\,\rvct{s})$ is
\begin{equation}
\begin{split}
  &\pden{\rvct{x},\rvct{s}}(\vct{x},\vct{s})
  =\,\\
  &\quad
  \frac{1}{(2\pi)^{\nicefrac{D}{2}} Z_B} \exp\lpa 
    -\frac{1}{2} \vct{x}\tr\vct{x} + \vct{s}\tr\mtx{Q}\vct{x} 
    -\frac{1}{2}\vct{s}\tr\mtx{Q}\mtx{Q}\tr\vct{s} + \frac{1}{2}\vct{s}\tr\mtx{W}\vct{s} 
    + \vct{s}\tr\vct{b}
  \rpa
\end{split}
\end{equation}
Using that $\mtx{Q}\mtx{Q}\tr = \mtx{W} + \mtx{D}$ this can be simplified to
\begin{align}
 \pden{\rvct{x},\rvct{s}}(\vct{x},\vct{s})
  &=
  \frac{1}{(2\pi)^{\nicefrac{D}{2}} Z_B} \exp\lpa 
    -\frac{1}{2} \vct{x}\tr\vct{x} + \vct{s}\tr\lpa \mtx{Q}\vct{x} + \vct{b} \rpa
    -\frac{1}{2}\vct{s}\tr\mtx{D}\vct{s}
  \rpa
  \\
  &=
  \frac{\exp\lpa -\frac{1}{2} \vct{x}\tr\vct{x} \rpa}{(2\pi)^{\nicefrac{D}{2}} Z_B \exp\lpa\frac{1}{2}\Tr(\mtx{D})\rpa} 
  \prod_{i=1}^{D_B} \lpa
  \exp\lpa
    s_i\lpa \vct{q}_i\tr\vct{x} + b_i \rpa
  \rpa
  \rpa,
\end{align}
where $\fset{\vct{q}_i\tr}_{i=1}^{D_B}$ are the $D_B$ rows of $\mtx{Q}$.
\noindent
We can marginalise over the binary state $\rvct{s}$ as each $\rvar{s}_i$ is conditionally independent of all the others given $\rvct{x}$ in the joint distribution. This gives the \emph{Boltzmann machine relaxation} density on $\rvct{x}$
\begin{equation}
  \pden{\rvct{x}}(\vct{x}) =
  \frac
  {2^{D_B} \exp\lpa -\frac{1}{2} \vct{x}\tr\vct{x} \rpa }
  {(2\pi)^{\nicefrac{D}{2}} Z_B \exp\lpa\frac{1}{2}\Tr(\mtx{D})\rpa} 
  \prod_{i=1}^{D_B} \lpa \cosh\lpa\vct{q}_i\tr\vct{x} + b_i \rpa\rpa,
\end{equation}
which is a structured Gaussian mixture density with $2^{D_B}$ components.
\noindent
If we define $\pden{\rvct{x}}(\vct{x}) = \frac{1}{Z} \exp\lpa-\phi(\vct{x})\rpa$ with
\begin{equation}
  \phi(\vct{x}) = 
  \frac{1}{2} \vct{x}\tr\vct{x} -
  \sum_{i=1}^{D_B} \lpa \log\cosh\lpa\vct{q}_i\tr\vct{x} + b_i \rpa\rpa,
\end{equation}
then the normalisation constant $Z$ of the relaxation density can be related to the normalising constant of the corresponding Boltzmann machine distribution by
\begin{equation}
  \log{Z} = \log Z_B + \frac{1}{2}\Tr(\mtx{D}) + \frac{D}{2}\log(2\pi) - D_B\log 2.
\end{equation}
\noindent
It can also be shown that the first and second moments of the relaxation distribution are related to the first and second moments of the corresponding Boltzmann machine distribution by
\begin{equation}
\begin{split}
  \expc{\rvct{x}} 
  &= \int_{\set{X}} \vct{x} \sum_{\vct{s}\in\set{S}} \lpa 
    \pden{\rvct{x}|\rvct{s}}(\vct{x} \gvn \vct{s}) \,\prob{\rvct{s}}(\vct{s})
  \rpa 
  \,\dr\vct{x} \\
  &= \sum_{\vct{s}\in\set{S}} \lpa 
    \int_{\set{X}} \vct{x} \,\nrm{\vct{x}; \mtx{Q}\tr\vct{s},\,\idmtx} \,\dr\vct{x}
    \,\prob{\rvct{s}}(\vct{s})
  \rpa \\
  &= \expc{
    \mtx{Q}\tr\rvct{s}
  }
  =
  \mtx{Q}\tr \expc{\rvct{s}},
\end{split}
\end{equation}
\begin{equation}
\begin{split}
  \expc{\rvct{x}\rvct{x}\tr} 
  &= \sum_{\vct{s}\in\set{S}} \lpa 
    \int_{\set{X}} \vct{x}\vct{x}\tr \,\nrm{\vct{x}; \mtx{Q}\tr\vct{s},\,\idmtx} \,\dr\vct{x}
    \,\prob{\rvct{s}}(\vct{s}) 
  \rpa\\
  &= \expc{\mtx{Q}\tr\rvct{s}\rvct{s}\mtx{Q} + \idmtx}
  = \mtx{Q}\tr\expc{\rvct{s}\rvct{s}\tr}\mtx{Q} + \idmtx.
\end{split}
\end{equation}