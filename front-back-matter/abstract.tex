\frontmattersection{Abstract}

Inference, the process of drawing conclusions from evidence, is at the heart of the scientific method. Probability theory offers a consistent framework for performing inference in the presence of uncertainty, posing the inference problem as the task of computing expectations of functions of interest with respect to a probability distribution.

%expectations with respect to a probability distribution conditioned on our assumptions and observations.

In complex models with a large number of unknown variables to be inferred, these expecations can be intractable to compute exactly. This has motivated the development of a large class of approximate inference methods which tradeoff a lack of exactness for greater computational tractability. Monte Carlo methods are one class of such techniques, with the integrals or summations across the whole state space approximated by summations over a finite number of randomly sampled points. This maps the inference problem to that of drawing samples from, often complex, high-dimesional, probability distributions.

%Lorem ipsum at nusquam appellantur his, ut eos erant homero concluda turque. Albucius appellantur deterruisset id eam, vivendum partiendo dissentiet ei ius. Vis melius facilisis ea, sea id convenire referrentur, takimata adolescens ex duo. Ei harum argumentum per. Eam vidit exerci appetere ad, ut vel zzril intellegam interpretaris. \marginpar{This is a test margin note. Foo bar lorem ipsum.}

%Errem omnium ea per, pro \ac{MCMC} con populo ornatus cu, ex qui dicant nemore melius. No pri diam iriure euismod. Graecis eleifend appellantur quo id. Id corpora inimicus nam, facer nonummy ne pro, kasd repudiandae ei mei. Mea menandri mediocrem dissentiet cu, ex nominati imperdiet nec, sea odio duis vocent ei. Tempor everti appareat cu ius, ridens audiam an qui, aliquid admodum conceptam ne qui. Vis ea melius nostrum, mel alienum euripidis eu.