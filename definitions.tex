\DeclareMathOperator{\expectation}{\vvmathbb{E}}
\DeclareMathOperator{\indicator}{\vvmathbb{1}}
\DeclareMathOperator*{\argmin}{arg\,min}
\DeclareMathOperator*{\argmax}{arg\,max}
\DeclareMathOperator{\chol}{chol}
\DeclareMathOperator{\gvn}{|}
\DeclareMathOperator{\Tr}{Tr}
\DeclareMathOperator{\trace}{Trace}
\DeclareMathOperator{\dimension}{dim}
\DeclareMathOperator{\diag}{diag}
\DeclareMathOperator{\sech}{sech}
\DeclareMathOperator{\cosech}{cosech}

\let\originalleft\left
\let\originalright\right
\renewcommand{\left}{\mathopen{}\mathclose\bgroup\originalleft}
\renewcommand{\right}{\aftergroup\egroup\originalright}

\newcommand{\probdens}{\mathsf{p}}
\newcommand{\probability}{\mathsf{P}}
%\newcommand{\probdens}{\vvmathbb{p}}
%\newcommand{\probability}{\vvmathbb{P}}
\newcommand{\pden}[1]{\probdens_{#1}}  
\newcommand{\prob}[1]{\probability_{#1}}   

\newcommand{\lebesgue}{\lambda}
\newcommand{\hausdorff}{\mathcal{H}}
\newcommand{\lebm}[1]{
  \ifthenelse { \equal {#1} {1} }
    { \lebesgue }   % if #1 == blank
    { \lebesgue^{#1} }   % else (not blank)
}
\newcommand{\haum}[1]{
  \ifthenelse { \equal {#1} {} }
    { \hausdorff }   % if #1 == blank
    { \hausdorff^{#1} }   % else (not blank)
}
\newcommand{\countm}{\#}

\newcommand{\borel}{\mathscr{B}}
\newcommand{\powerset}{\mathscr{P}}

\newcommand{\vct}[1]{\boldsymbol{#1}}
\newcommand{\mtx}[1]{\boldsymbol{#1}}
%\newcommand{\set}[1]{\mathcal{#1}}
%\newcommand{\sset}[1]{\mathscr{#1}}
\newcommand{\set}[1]{\mathit{#1}}
\newcommand{\sset}[1]{\mathcal{#1}}
\newcommand{\fset}[1]{\lbr #1 \rbr}
\newcommand{\reals}{\vvmathbb{R}}
\newcommand{\naturals}{\vvmathbb{N}}
\newcommand{\integers}{\vvmathbb{Z}}
\newcommand{\ind}[1]{\indicator_{#1}}
\newcommand{\bigo}[1]{\mathcal{O}\lpa #1\rpa}
\newcommand{\lpa}{\left(}
\newcommand{\rpa}{\right)}
\newcommand{\lbr}{\left\lbrace}
\newcommand{\rbr}{\right\rbrace}
\newcommand{\lsb}{\left[}
\newcommand{\rsb}{\right]}

%\newcommand{\prob}[1]{\vvmathbb{P}\lpa #1 \rpa}
%\newcommand{\prob}[2]{\probability_{#1}\lpa #2 \rpa}
%\newcommand{\probnoarg}[1]{\probability_{#1}}
%\newcommand{\pden}[1]{\vvmathbb{p}\lpa #1 \rpa}
%\newcommand{\pden}[2]{\probdens_{#1}\lpa #2 \rpa}
%\newcommand{\pdennoarg}[1]{\probdens_{#1}}
%\newcommand{\rvar}[1]{%
%  \mathchoice
%  {\mbox{$\mathit{\expandafter\MakeUppercase\expandafter{#1}}$}}
%  {\mbox{\small$\mathit{\expandafter\MakeUppercase\expandafter{#1}}$}}
%  {\mbox{\tiny$\mathit{\expandafter\MakeUppercase\expandafter{#1}}$}}
%  {\mbox{\tiny$\mathit{\expandafter\MakeUppercase\expandafter{#1}}$}}
%}
\newcommand{\rvar}[1]{\mathsf{#1}}
%\newcommand{\rvar}[1]{\mathrm{#1}}
\newcommand{\rvct}[1]{\boldsymbol{\rvar{#1}}}
\newcommand{\nrm}[1]{\mathcal{N}\lpa #1 \rpa}
\newcommand{\rng}[2]{\lbr #1 \dots #2 \rbr}
\newcommand{\expc}[1]{\vvmathbb{E}\lsb #1 \rsb}
\newcommand{\var}[1]{\vvmathbb{V}\lsb #1 \rsb}
\newcommand{\dr}{\mathrm{d}}
\newcommand{\td}[2]{\frac{\dr #1}{\dr #2}}
\newcommand{\tdd}[2]{\frac{\dr^2 #1}{\dr #2^2}}
\newcommand{\pd}[2]{\frac{\partial #1}{\partial #2}}
\newcommand{\pdd}[3]{\frac{\partial^2 #1}{\partial #2 \partial #3}}
\newcommand{\tr}{^{\mkern-1.5mu\mathsf{T}}}
\newcommand{\sml}[1]{#1}
\newcommand{\obs}[1]{\breve{#1}}
\newcommand{\hamiltonian}{h}
\newcommand{\extended}[1]{\skew{2}{\tilde}{#1}}
\newcommand{\normconst}{Z}
\newcommand{\partfunc}{z}
\newcommand{\transition}{\mathbb{T}}

\newcommand{\func}[1]{\mathrm{#1}}
\newcommand{\vctfunc}[1]{{\mathbf{#1}}}

\newcommand{\inlinetr}{\phantom{}^\textsc{t}}

\makeatletter % Define relationships between lower and upper case Greek characters
  \g@addto@macro\@uclclist{%
    \eth\Eth
    \thorn\Thorn
    \alpha\Alpha
    \beta\Beta
    \gamma\Gamma
    \delta\Delta
    \epsilon\Epsilon
    \varepsilon\Varepsilon
    \zeta\Zeta
    \eta\Eta
    \theta\Theta
    \vartheta\Vartheta
    \iota\Iota
    \kappa\Kappa
    \lambda\Lambda
    \mu\Mu
    \nu\Nu
    \xi\Xi
    \omicron\Omicron
    \pi\Pi
    \varpi\Varpi
    \rho\Rho
    \varrho\Varrho
    \sigma\Sigma
    \varsigma\Varsigma
    \tau\Tau
    \upsilon\Upsilon
    \phi\Phi
    \varphi\Varphi
    \chi\Chi
    \psi\Psi
    \omega\Omega
  }

\usepackage{scalerel}
\newsavebox{\foobox}
\newcommand{\slantbox}[2][0]{\colorlet{slantcolor}{.}\mbox{%
        \sbox{\foobox}{\color{slantcolor}#2}%
        \hskip\wd\foobox
        \pdfsave
        \pdfsetmatrix{1 0 #1 1}%
        \llap{\usebox{\foobox}}%
        \pdfrestore
}}
\newcommand\unslant[2][-.2]{%
  \mkern1mu%
  \ThisStyle{\slantbox[#1]{$\SavedStyle#2$}}%
  \mkern-1mu%
}

%\DeclareSymbolFont{sfletters}{U}{eur}{m}{n}
%\SetSymbolFont{sfletters}{sans}{U}{eur}{m}{n}

\DeclareSymbolFont{sfletters}{OML}{cmssm}{m}{it}

\DeclareMathSymbol{\sfalpha}{\mathord}{sfletters}{"0B}
\DeclareMathSymbol{\sfbeta}{\mathord}{sfletters}{"0C}
\DeclareMathSymbol{\sfgamma}{\mathord}{sfletters}{"0D}
\DeclareMathSymbol{\sfdelta}{\mathord}{sfletters}{"0E}
\DeclareMathSymbol{\sfepsilon}{\mathord}{sfletters}{"0F}
\DeclareMathSymbol{\sfzeta}{\mathord}{sfletters}{"10}
\DeclareMathSymbol{\sfeta}{\mathord}{sfletters}{"11}
\DeclareMathSymbol{\sftheta}{\mathord}{sfletters}{"12}
\DeclareMathSymbol{\sfiota}{\mathord}{sfletters}{"13}
\DeclareMathSymbol{\sfkappa}{\mathord}{sfletters}{"14}
\DeclareMathSymbol{\sflambda}{\mathord}{sfletters}{"15}
\DeclareMathSymbol{\sfmu}{\mathord}{sfletters}{"16}
\DeclareMathSymbol{\sfnu}{\mathord}{sfletters}{"17}
\DeclareMathSymbol{\sfxi}{\mathord}{sfletters}{"18}
\DeclareMathSymbol{\sfpi}{\mathord}{sfletters}{"19}
\DeclareMathSymbol{\sfrho}{\mathord}{sfletters}{"1A}
\DeclareMathSymbol{\sfsigma}{\mathord}{sfletters}{"1B}
\DeclareMathSymbol{\sftau}{\mathord}{sfletters}{"1C}
\DeclareMathSymbol{\sfupsilon}{\mathord}{sfletters}{"1D}
\DeclareMathSymbol{\sfphi}{\mathord}{sfletters}{"1E}
\DeclareMathSymbol{\sfchi}{\mathord}{sfletters}{"1F}
\DeclareMathSymbol{\sfpsi}{\mathord}{sfletters}{"20}
\DeclareMathSymbol{\sfomega}{\mathord}{sfletters}{"21}
\DeclareMathSymbol{\sfvarepsilon}{\mathord}{sfletters}{"22}
\DeclareMathSymbol{\sfvartheta}{\mathord}{sfletters}{"23}
\DeclareMathSymbol{\sfvarpi}{\mathord}{sfletters}{"24}
\DeclareMathSymbol{\sfvarrho}{\mathord}{sfletters}{"25}
\DeclareMathSymbol{\sfvarsigma}{\mathord}{sfletters}{"26}
\DeclareMathSymbol{\sfvarphi}{\mathord}{sfletters}{"27}

\newcommand{\upsfalpha}{\unslant\sfalpha}